\documentclass[letterpaper,10pt]{article}
\usepackage[utf8]{inputenc}
\usepackage{amsmath}
\usepackage{amssymb}
\usepackage[italian]{babel}
\usepackage{anyfontsize}
\usepackage{hyperref}
\usepackage{nicefrac}
\usepackage{tikz}
\usepackage{pgfplots}
\usepackage{caption}
\usepackage{graphicx}
\usepackage{subcaption}
\usetikzlibrary{patterns,plotmarks}
\usepackage{multirow}
\usepackage[
backend=bibtex,
style=numeric,
sorting=ynt
]{biblatex}
\addbibresource{biblio.bib}

\newcommand{\Zn}{$\mathbb{Z}^n_k$}
\newcommand{\Zii}{$\mathbb{Z}^2_{31}$}


\begin{document}

\title{\Huge Simulazione di Dinamica di Opinione su Automa Cellulare 
		\[\]\Large Relazione scritta per il corso di \\ Introduzione alla fisica dei sistemi complessi \medskip
		}
\author{Matteo Santini, Matteo Costa}
\date{13 Maggio 2022}

\maketitle
\tableofcontents
\bigskip

 \section{Abstract}
Il presente progetto si occupa di analizzare il comportamento di una popolazione in un problema di \textit{opinion dynamics} tramite la modellizzazione e la realizzazione simulativa di un automa cellulare. A questo scopo è stato conveniente utilizzare il modello di Ising 2D, opportunamente adattato, per gestire l'evoluzione temporale delle opinioni di individui all'interno di una comunità chiusa. In particolare, è stata studiata la risposta del sistema a tre diverse condizioni ambientali. Si è ritrovato che in generale, per una comunità di questo tipo, la distribuzione delle opinioni tende a rilassare verso due condizioni stazionarie in modalità e tempistiche differenti a seconda del caso in esame.

\section{Introduzione}
\label{Sec:2}
Questo progetto di carattere simulativo è finalizzato allo studio statistico dell'evoluzione temporale di una popolazione di individui dotati di una opinione personale. 
\\ A questo scopo, si decidono di introdurre alcune importanti approssimazioni, cercando di mantenersi sempre il più possibile realistici. Ad esempio, si riduce il parametro di opinione ad una mera variabile discreta a due soli valori possibili ($\pm$1), introducendo così una sorta di analogia al Modello di Ising 2D \cite{ising}, che in meccanica statistica permette la descrizione di interazioni locali tra spin nel reticolo cristallino di un materiale condensato per spiegarne l'insorgenza di proprietà magnetiche tramite transizioni di fase interne.
\\ Questo tipo di modello, infatti, si addice alla descrizione di numerosi sistemi fisici complessi, dove è possibile osservare fenomeni di auto-organizzazione, e quello trattato qui di seguito ne è un chiaro esempio. 
\\ Tuttavia, è però necessario avere l'accortezza di adattare le regole di evoluzione canoniche, in quanto a differenza degli atomi nei reticoli cristallini, su cui si è soliti applicare il Modello di Ising, gli individui sono in grado di compiere movimenti in modo simultaneo ad ogni step temporale. Dunque, per introdurre interazioni locali tra gli individui della popolazione si sceglie di definire un parametro di distanza d'influenza, che permetta di selezionare il numero finito di altri individui da cui ciascuno viene influenzato in termini di opinione.
\\ Successivamente, è stata studiata la reazione del sistema in esame a due particolari codizioni, che a seguire andranno sotto il nome di visione parziale e \textit{flocking} gravitazionale.
\\ La prima consiste sostanzialmente nell'ipotesi di limitare il numero di individui da cui ciascuno viene influenzato ad un valore costante nel tempo ed uniforme sull'intera popolazione. Ciò, infatti, risulta essere un'interessante tema da approfondire, notando che mediamente nella società le singole persone manifestano, più o meno consapevolmente, una propensione ad interagire significativamente con solo poche delle proprie conoscenze effettive.
\[\]
\\ La seconda e ultima, invece, introduce nella dinamica del sistema una sorta di campo gravitazionale che implementi una attratività reciproca tra individui di medesima opinione, simulando così il fenomeno di formazione di comunità omogenee nella società.


\section{Modello Matematico}
\label{Sec:3}

\subsection{Automa Cellulare}
\label{Sec:3.1}
Il seguente modello ad automa cellulare \cite{modeling} intende porre le basi teoriche matematiche necessarie all'implementazione di una simulazione del comportamento di una popolazione all’interno di uno spazio bidimensionale discretizzato, dato da tutte le coppie di numeri interi nel prodotto cartesiano $[1,k+1]\times[1,k+1]$ con $k=31$.

\begin{figure}[h]
\centering
\begin{tikzpicture}[
  scale=0.95,
  mydot/.style={
    circle,
    fill=white,
    draw,
    outer sep=0pt,
    inner sep=1pt
  }
]
\draw[step=0.2cm,gray!50!white,very thin] (0,0) grid (6.2,6.2);
\draw[thick,->] (0,0) -- (7,0) node[anchor=north west] {x};
\draw[thick,->] (0,0) -- (0,7) node[anchor=south east] {y};
\foreach \x in {0,5,10,15,20,25,30}
   \draw (\x*0.2 cm,2pt) -- (\x*0.2 cm,-2pt) node[anchor=north] {$\x$};
\foreach \y in {0,5,10,15,20,25,30}
    \draw (2pt,\y*0.2 cm) -- (-2pt,\y*0.2 cm) node[anchor=east] {$\y$};
\filldraw[fill=red!40!red, draw=black] (4,2) circle (0.1cm);
\end{tikzpicture}
\caption{\textit{Rappresentazione grafica dello spazio bidimensionale discretizzato, su cui è disposto arbitrariamente un individuo evidenziato in rosso.}}
\label{Fig:1}
\end{figure}

Come è possibile osservare in Figura \ref{Fig:1}, tali coppie compongono le coordinate di tutti i possibili siti dell'automa cellulare, che possono o meno essere occupati dagli individui.
\\ La popolazione in esame è costituita da un numero costante \textit{N=100} di individui, anche detti agenti, i quali sono caratterizzati fondamentalmente da un parametro di opinione personale, che ne definisce univocamente lo stato.
In particolare, questo può assumere solo i valori interi $\pm$1, i quali sono interpretati come l'assenso o il dissenso verso un certo tema politico-sociale.
\bigskip \bigskip
\\ Inoltre, come anticipato in Sezione~\ref{Sec:2}, ad ogni istante dell’evoluzione temporale del sistema, gli agenti sono in grado di compiere in modo simultaneo movimenti all'interno del proprio intorno di Von Neumann di raggio unitario, come rappresentato in Figura \ref{Fig:2}.

\begin{figure}[h]
\centering
\begin{tikzpicture}[
  scale=0.9,
  mydot/.style={
    circle,
    fill=white,
    draw,
    outer sep=0pt,
    inner sep=1pt
  }
]
\draw[step=1cm,gray!50!white,very thin] (0.5,0.5) grid (4.5,4.5);
\fill[fill=green!30!white] (2,2) rectangle (3,3);
\filldraw[fill=green!30!white, draw=black] (2,1) rectangle (3,2);
\filldraw[fill=green!30!white, draw=black] (1,2) rectangle (2,3);
\filldraw[fill=green!30!white, draw=black] (2,3) rectangle (3,4);
\filldraw[fill=green!30!white, draw=black] (3,2) rectangle (4,3);
\filldraw[fill=red!40!red, draw=black] (2.5,2.5) circle (0.5cm);
\put(60,85){$N$};
\put(60,35){$S$};
\put(35,60){$O$};
\put(85,60){$E$};
\end{tikzpicture}
\caption{\textit{Visualizzazione di tutti i siti (in verde) appartenenti all'intorno di Von Neumann di un individuo (in rosso) con relative direzioni (Nord, Sud, Est, Ovest).}}
\label{Fig:2}
\end{figure}

Da notare come in realtà sia contemplata anche la possibilità di rimanere fermi, ma limitatamente al caso in cui l'individuo si trovasse occupate tutte le celle nel proprio intorno. Ciò si è reso necessario affinchè in generale sia prediletta la scelta di compiere un movimento rispetto a quella di bloccarsi.
\\ L'effettiva direzione degli spostamenti è scelta stocasticamente tra le quattro permesse (Nord, Sud, Est, Ovest + Stessa posizione nel caso particolare precedentemente descritto), attribuendo a ciascuna di esse una medesima probabilità $\nicefrac{1}{n}$ con n pari al numero di siti liberi adiacenti, tranne quando questo si annulla, in cui invece si permette all’individuo di mantenere la propria posizione, garantendo così sempre la normalizzazione ad 1.
\\ Tali spostamenti all’interno dello spazio discretizzato bidimensionale (0,31]x(0,31] sono vincolati a condizioni al contorno periodiche. Ne segue dunque che topologicamente l’insieme dei siti dell’automa cellulare costituisce una struttura toroidale, esplicitata in Figura \ref{Fig:3} e d'ora in poi indicata con \Zn, dove n=2 e k=31 sono rispettivamente le dimensioni spaziali ed il numero di siti per lato dell'automa cellulare.
\\ Questa scelta implementativa è espressamente motivata dall’obiettivo di evitare moti attrattivi degli individui verso ai bordi, che andrebbero così ad influenzare non trascurabilmente l’evoluzione del sistema.
\\ La dinamica del sistema, di fatto, prevede la scelta del valore di un parametro di controllo, ossia $\delta$ definito come il raggio di influenza, e limitato agli elementi del seguente insieme \{0,...,$\nicefrac{(31-1)}{2}=15$\}. In base al suo valore si decide che l'individuo arbitrario \textit{i} $\in$ $\{1,...,\textit{N} \}$ può interagire solo con coloro la cui posizione è compresa nel suo intorno di influenza $\Delta_{i}(t)$ = $\{ x \in \ $\Zii$ \ | \  \ d(x_{i}(t),x)\leqslant \delta \}$ con \textit{$x_{i}$(t)} come posizione dell'individuo \textit{i} all'istante \textit{t}, mentre \textit{d(x,y)} come metrica di Chebyshev nello spazio discrezzato bidimensionale dell'automa cellulare. 

\begin{figure}[h]
\centering
\begin{tikzpicture}
\begin{axis} 
[view={20}{55}, title={Periodic Boundary Conditions}, ytick=\empty, xtick=\empty, ztick=\empty, axis line style={draw=none}]
\addplot3
[domain=0:360, y domain=0:360, variable=\u, variable y=\v, mesh, samples=30, z buffer=sort, surf, colormap/blackwhite]
({(3+cos(u))*cos(v)}, {(3+cos(u))*sin(v)}, {sin(u)});
\filldraw[fill=red!40!red, draw=black] (420,210) circle (0.05cm);
\filldraw[fill=red!40!red, draw=black] (200,450) circle (0.05cm);
\filldraw[fill=red!40!red, draw=black] (590,690) circle (0.05cm);
\filldraw[fill=red!40!red, draw=black] (300,550) circle (0.05cm);
\filldraw[fill=blue!40!blue, draw=black] (620,800) circle (0.05cm);
\filldraw[fill=blue!40!blue, draw=black] (455,250) circle (0.05cm);
\filldraw[fill=blue!40!blue, draw=black] (400,900) circle (0.05cm);
\filldraw[fill=red!40!red, draw=black] (380,800) circle (0.05cm);
\filldraw[fill=blue!40!blue, draw=black] (150,680) circle (0.05cm);
\filldraw[fill=blue!40!blue, draw=black] (0,700) circle (0.05cm);
\filldraw[fill=blue!40!blue, draw=black] (400,700) circle (0.05cm);
\filldraw[fill=red!40!red, draw=black] (100,1000) circle (0.05cm);
\end{axis}
\end{tikzpicture}
\caption{\textit{Struttura toroidale dell'automa cellulare dovuta alla condizioni al contorno periodiche.}}
\label{Fig:3}
\end{figure}

\bigskip \bigskip
Da notare come la validità delle condizioni al contorno periodiche comporta la costruzione di range d'influenza, anche detti intorni di Moore estesi, molto particolari, di cui sono riportati alcuni esempi in Figura \ref{Fig:4}.

\begin{figure}[h]
\centering
\begin{tikzpicture}[
  scale=0.7,
  mydot/.style={
    circle,
    fill=white,
    draw,
    outer sep=0pt,
    inner sep=1pt
  }
]
\fill[fill=green!30!white] (1,2.5) rectangle (3.5,5);
\fill[fill=green!30!white] (7,0) rectangle (8.5,2);
\fill[fill=green!30!white] (7,5.5) rectangle (8.5,6);
\fill[fill=green!30!white] (12,0) rectangle (13,2);
\fill[fill=green!30!white] (12,5.5) rectangle (13,6);
\draw[step=0.5cm,gray!50!white,very thin] (-1,0) grid (5,6);
\draw[step=0.5cm,gray!50!white,very thin] (6.99,0) grid (13,6);
\filldraw[fill=red!40!red, draw=black] (2.25,3.75) circle (0.25cm);
\filldraw[fill=red!40!red, draw=black] (7.25,0.75) circle (0.25cm);
\end{tikzpicture}
\caption{\textit{Intorni di Moore estesi di raggio 2 in due differenti casi.}}
\label{Fig:4}
\end{figure}

In questi termini, una volta noto lo stato del sistema dato dalla conoscenza della corrispondenza $S(t): i \rightarrow s(i,t), $ è possibile quantificare l'influenza a cui un individuo generico $\textit{i} \in \{1,...,\textit{N} \}$ è soggetto all'istante temporale \textit{t} come il seguente parametro reale nell'intervallo compatto $[-1,+1]$

\begin{equation}
    \sigma(i,t) = \sum_{ \{ j \ | \ x_{j} \in \Delta_{i} \} } s(j, t)p(M_{i}(t))
    \medskip
    \label{Eq:1}
\end{equation}

che appunto consiste di una media uniformemente pesata degli stati, ossia delle opinioni, \textit{s(j,t)} degli individui \textit{j} la cui posizione è compresa nell'intorno d'influenza $\Delta_{i}(t)$ precedentemente definito all'istante \textit{t}. Infatti, i coefficienti moltiplicativi \textit{p}($M_{i}(t)$) = $\nicefrac{1}{M_{i}(t)}$, con $M_{i}(t)$ pari al numero di agenti influenti su \textit{i} all'istante \textit{t}, non dipendono dall'indice \textit{j} su cui viene eseguita l'operazione di sommatoria e soddisfano la condizione di normalizzazione:

\begin{equation}
    \sum_{ \{ j \ | \ x_{j} \in \Delta_{i} \} } p(M_{i}(t)) = M_{i}(t) \cdot p(M_{i}(t)) =    M_{i}(t) \cdot \left ( \frac{1}{M_{i}(t)} \right ) = 1.
    \medskip
    \label{Eq:2}
\end{equation}

Per finire, si implementa l'evoluzione temporale locale dell'opinione dell'agente \textit{i} valutando l'influenza associata $\sigma(i,t)$  calcolata in Equazione \ref{Eq:1} tramite la funzione a valori interi $I$ che ne caratterizza l'andamento ed è così definita

\begin{equation}
    s(i,t+1) = I(\sigma(i,t)) = sign(\sigma(i,t)).
    \medskip
    \label{Eq:3}
\end{equation}

La scelta di tale regola di evoluzione del sistema determina una forte sensibilità degli individui all'influenza percepita, perchè non appena questa assume un segno discorde dalla loro opinione si verifica una transizione di stato \mbox{(+1 $\longleftrightarrow$ -1)} nonostante il suo modulo risulti essere $|\sigma(i,t)| \ll 1$.
\\ Al fine, quindi, di rallentare la dinamica delle opinioni, è stato introdotto un ulteriore parametro \textit{T}, definito positivo e analogo alla temperatura nei sistemi fisici, che moduli la velocità di rilassamento alla distribuzione stazionaria in termini di probabilità di transizione di stato $ \pi_{\pm}(T) = \pi(T) = \tanh (T)$.
\\ Per considerazioni di convenienza computazionale, è stato scelto un valore costante $T=1$.

\bigskip
\begin{figure}[h]
\centering
\begin{tikzpicture}[line cap=round, line join=round,
                            x=3.0cm, y=2.0cm, scale=1]
\draw [->,color=black, line width=1.0pt] (-0.05,0) -- (2.5,0);
\foreach \x in {1}
\draw [shift={(\x,0)}, color=black] (0pt,2pt)
-- (0pt,-2pt) node [below] {$\x$};
\draw [color=black] (2.5,0) node [anchor= north west] {$T$};
\draw [->,color=black, line width=1.0pt] (0,-0.1) -- (0,2);
\foreach \y in {0.7616,1}
\draw [shift={(0,\y)}, color=black] (2pt,0pt)
-- (-2pt,0pt) node[left] {$\y$};
\draw [color=black] (0,2) node [anchor= south east] {$\pi (T)$};
\draw [color=black] (0pt,-10pt) node [left] {$0$};
\draw [domain=0:2.4, line width=1.0pt, color=blue] plot (\x,{tanh(\x)});
\draw [dashed, color=gray, line width=0.6pt] (1,0) -- (1,1);
\draw [dashed, color=gray, line width=0.6pt] (0,1) -- (2.45,1);
\draw [dashed, color=gray, line width=0.6pt] (0,0.7616) -- (1,0.7616);
\filldraw[fill=white, draw=blue, line width=0.8pt] (1,0.7616) circle (0.08cm);
\end{tikzpicture}
\caption{\textit{Andamento caratteristico della probabilità di transizione di stato al variare della temperatura T, con il valore scelto per le simulazioni in evidenza.}}
\label{Fig:5}
\end{figure}

\bigskip \bigskip
\subsection{Visione Parziale}
\label{Sec:3.2}

Come anticipato in Sezione \ref{Sec:2}, il sistema appena descritto viene sottoposto ad una condizione esterna globale, detta visione parziale. Questa prevede l'introduzione di un ulteriore parametro di controllo discreto $\eta$ e la scelta di un valore fissato arbitrario di distanza d'influenza (ad esempio, nelle simulazioni $\delta=4$).
\\ Il parametro $\eta$ decide il numero massimo di individui da cui ciascuno può essere influenzato, ed è costante nel tempo, uniforme su tutta la popolazione e limitato agli elementi del seguente insieme $\{ 1, ..., (2\delta+1)(2\delta+1)-1=80\}$, come si può dedurre da Figura \ref{Fig:6}.

\begin{figure}[h]
\centering
\begin{tikzpicture}[
  scale=0.95,
  mydot/.style={
    circle,
    fill=white,
    draw,
    outer sep=0pt,
    inner sep=1pt
  }
]
\fill[fill=green!30!white] (1,1) rectangle (2.8,2.8);
\draw[step=0.2cm,gray!50!white,very thin] (0,0) grid (4,4);
\draw[thick,->] (1,4.1) -- (2.8,4.1) node[] {};
\draw[thick,->] (2.8,4.1) -- (1,4.1) node[] {};
\draw[thick,->] (4.1,1) -- (4.1,2.8) node[] {};
\draw[thick,->] (4.1,2.8) -- (4.1,1) node[] {};
\filldraw[fill=red!40!red, draw=black] (1.9,1.9) circle (0.1cm);
\put(35,118){$(2\delta+1)$};
\put(115,48){$(2\delta+1)$};
\end{tikzpicture}
\caption{\textit{Rappresentazione grafica dell'intorno d'influenza $\Delta$ di un individuo, posizionato arbitrariamente, con $\delta$=4. Utile per comprendere che necessariamente il limite superiore di $\eta$ debba essere il numero di siti in $\Delta$, escluso quello occupato dall'individuo stesso.}}
\label{Fig:6}
\end{figure}

Oltre a questa peculiaretà, la dinamica e l'evoluzione temporale delle opinioni seguono sostanzialmente le medesime regole valide per l'automa cellulare, specificate in Sezione \ref{Sec:3.1}.


\subsection{Flocking Gravitazionale}
\label{Sec:3.3}

In questo modello ad automa cellulare sono stati inseriti due campi che agiscono su tutti gli individui della stessa opinione attraverso una forza attrattiva proporzionale a $\nicefrac{1}{r^2}$. Fisicamente questi campi corrispondono ad una coppia di campi gravitazionali

\begin{equation}
\begin{aligned}
\begin{cases}
F_x:  \mathbb{Z}^2_k \rightarrow \mathbb{R}^2 \\
F_y:  \mathbb{Z}^2_k \rightarrow \mathbb{R}^2 
\end{cases}
\end{aligned}
\label{Eq:4}
\end{equation}

la cui forza agente tra due individui \textit{i}, \textit{j} a distanza $\vec{r}_{x}$ lungo le ascisse e $\vec{r}_{y}$ lungo le ordinate, aventi la medesima opinione è

\begin{equation}
\begin{aligned}
    \vec{F}_x^{(i,j)} =& \dfrac{Gm_im_j}{r_x^2} \hat{r_x} \quad\quad con \quad i\neq j\\
    \vec{F}_y^{(i,j)} =& \dfrac{Gm_im_j}{r_y^2} \hat{r_y} \quad\quad con \quad i\neq j
\end{aligned}
\label{Eq:5}
\end{equation}

nell'ipotesi in cui tutti gli individui abbiano massa m=1 e che la costante gravitazionale sia G=10, per facilitare l'implementazione del modello (vedi Figura \ref{Fig:7}). 

\begin{figure}[h]
\centering
\begin{tikzpicture}[
  scale=0.7,
  mydot/.style={
    circle,
    fill=white,
    draw,
    outer sep=0pt,
    inner sep=1pt
  }
]
\draw[step=0.3cm,gray!50!white,very thin] (-1,0) grid (5,6);
\filldraw[fill=red!40!red, draw=black] (1.95,3.15) circle (0.1cm);
\draw [-latex] (2.05, 3.15) -- (4.05, 3.15);
\draw [-latex] (1.85, 3.15) -- (-0.15, 3.15);
\draw [-latex] (1.95, 3.25) -- (1.95, 5.25);
\draw [-latex] (1.95, 3.05) -- (1.95, 1.15);
\draw (0.65,5.2) node[] {} node[right] {$+F_y$};
\draw (1.95,1.2) node[] {} node[right] {$-F_y$};
\draw (-0.5,2.75) node[] {} node[right] {$-F_x$};
\draw (3.75,3.5) node[] {} node[right] {$+F_x$};
\end{tikzpicture}
\caption{\textit{Rappresentazione grafica delle possibili combinazioni di coppie di forze agenti su un individuo.}}
\label{Fig:7}
\end{figure}

L'introduzione di campi di questo tipo favorisce il fenomeno del \textit{flocking}, ossia la formazione di assembramenti tra individui della stessa opinione. Questo fenomeno introduce degli attrattori nella dinamica degli individui nello spazio \Zii: come si osserva in una comunità reale, persone aventi opinione politica affine tendono a stare assieme mentre, persone immerse in un ambiente in cui la propria ideologia politica non è ben vista, tendono a spostarsi dove ci sono individui che condividono la propria opinione.
\\
Le condizioni al contorno periodiche, che determinano la struttura toroidale dell'automa cellulare come spiegato in Sezione \ref{Sec:3.1}, possono rappresentare un difficoltà aggiuntiva ai fini della dinamica. Infatti, per ognuna delle due forze agenti tra due individui ci sono sempre due possibili distanze, ma è necessario che il modulo, direzione e verso di queste forze siano univocamente definiti per ogni coppia. Ad esempio, se consideriamo solo le forze agenti in direzione x, per ogni coppia di individui \textit{i}, \textit{j}, esiste una distanza che segue il verso positivo dell'asse delle ascisse, indicata con $r_x^+$, e una che va in senso opposto, denominata invece $r_x^-$. Assumendo una considerazione del tutto analoga anche per le forze agenti in y, le distanze utilizzate per i due campi vettoriali $\vec{F}^{(i,j)}_x$ e $\vec{F}^{(i,j)}_y$ sono state calcolate in base alle seguenti definizioni: rispettivamente,

\begin{equation}
\begin{aligned} 
&r_x(i,j) = min\{|r_x^+|, |r_x^-|\}\\
&r_y(i,j) = min\{|r_y^+|, |r_y^-|\}.
\end{aligned}
\label{Eq:6}
\end{equation}

Chiaramente, è necessario imporre delle ulteriori condizioni sulle forze per evitare dei punti singolari. Innanzitutto, si assume che se $r_x(i,j)$ o $r_y(i,j)$ si annulla, allora la forza corrispondente in x o y sarà nulla. È anche evidente che non è possibile avere entrambe $r_x(i,j)$ e $r_y(i,j)$ uguali a zero, perchè questo va in contraddizione con la richiesta $i\neq j$ (gli individui non si possono compenetrare). 
\\Allora, viene subito da domandarsi quando $\vec{F}^{(i,j)}_x$ e $\vec{F}^{(i,j)}_y$ possano essere entrambe nulle: questo caso particolare richiede delle considerazioni di carattere geometrico che risiedono nella struttura del toroide topologico. Quando, infatti, si realizza la condizione $r_x(i,j) = min\{|r_x^+|, |r_x^-|\} = |r_x^+| = |r_x^-|$ (ovvero quando le due distanze coincidono), si richiede che $\vec{F}^{(i,j)}_x$ = 0. Analogamente, ciò vale anche per $\vec{F}^{(i,j)}_y$, quando $r_y(i,j) = min\{|r_y^+|, |r_y^-|\} = |r_y^+| = |r_y^-|$. Ne è un chiaro esempio la configurazione in Figura \ref{Fig:8}.

\begin{figure}[hb]
\centering
\begin{subfigure}[h]{0.48\textwidth}
\centering
\begin{tikzpicture}[
  scale=0.7,
  mydot/.style={
    circle,
    fill=white,
    draw,
    outer sep=0pt,
    inner sep=1pt
  }
]
\draw[step=0.3cm,gray!50!white,very thin] (-1,0) grid (5,6);
\filldraw[fill=red!40!red, draw=black] (2.25,3.15) circle (0.1cm);
\filldraw[fill=red!40!red, draw=black] (-0.75,3.15) circle (0.1cm);
\end{tikzpicture}
\end{subfigure}
\hfill
\begin{subfigure}[h]{0.48\textwidth}
         \centering
\begin{tikzpicture}
\begin{axis} 
[view={0}{100}, ytick=\empty, xtick=\empty, ztick=\empty, axis line style={draw=none}]
\addplot3
[domain=0:360, y domain=0:360, variable=\u, variable y=\v, mesh, samples=30, z buffer=sort, surf, colormap/blackwhite]
({(3+cos(u))*cos(v)}, {(3+cos(u))*sin(v)}, {sin(u)});
\filldraw[fill=red!40!red, draw=black] (90,320) circle (0.06cm);
\filldraw[fill=red!40!red, draw=black] (710,300) circle (0.06cm);
\end{axis}
\end{tikzpicture}
\end{subfigure}
\caption{\textit{Esempio in cui la forza agente tra i due individui in rosso è nulla. A sinistra la rappresentazione nell'automa cellulare di questa situazione, a destra la sua corrispondenza sul toroide \Zii  che tiene conto delle periodic boundary conditions.}}
\label{Fig:8}
\end{figure}

Una volta determinati i contributi attrattivi $F_x^{(i,j)}(t)$ e $F_y^{(i,j)}(t)$ dovuti a tutti gli accoppiamenti possibili dell'individuo arbitrario \textit{i} con i suoi simili \textit{j} nella popolazione, si può passare alla stima delle forze gravitazionali totali $F_x^{(i)}(t)$ e $F_y^{(i)}(t)$ agenti proprio su \textit{i} all'istante temporale \textit{t} tramite le seguenti somme vettoriali

\begin{equation}
\begin{aligned}
    \vec{F}_x^{(i)}(t) = \sum_{ \{ j \ | \ s(j,t)=s(i,t) \} } \vec{F}_x^{(i,j)}(t) \\
    \vec{F}_y^{(i)}(t) = \sum_{ \{ j \ | \ s(j,t)=s(i,t) \} } \vec{F}_y^{(i,j)}(t).
\end{aligned}
\label{Eq:7}
\end{equation}

Sono questi vettori, dunque, i veri responsabili della particolare dinamica di \textit{flocking} gravitazionale.
\\ In base al loro modulo, direzione e verso vengono modificate le probabilità di scelta della direzione di spostamento dei singoli individui all'interno del proprio intorno unitario di Von Neumann, minimizzando progressivamente la distanza tra gli individui di egual opinione, in modo da formare delle regioni ad alta densità di individui e regioni a bassa densità di individui. 
\\ In particolare, la distribuzione di probabilità nella scelta della direzione degli spostamenti di ciascun individuo non sarà più uniforme, come descritto in Sezione \ref{Sec:3.1} per l'automa cellulare "imperturbato", ma polarizzata verso le direzioni parallele e concordi alle forze risultanti in modo tale da rispettare sempre la normalizzazione ad 1.
\\ Ciò è stato realizzato riscalando la crescita o decrescita di probabilità di scelta di una certa direzione rispetto al modulo quadro della forza totale risultante, ossia $|\vec{F}_{TOT}^{(i)}(t)|^2 =|\vec{F}_x^{(i)}(t)|^2+|\vec{F}_y^{(i)}(t)|^2$, nel seguente modo:

\begin{equation}
\begin{aligned}
\begin{cases}
P_x^{(i, \pm)}(t) = \dfrac{1}{4} \left ( 1 \pm \dfrac{|\vec{F}_x^{(i)}(t)|^2}{|\vec{F}_{TOT}^{(i)}(t)|^2} \right ) \\ \\
P_y^{(i, \pm)}(t) = \dfrac{1}{4} \left ( 1 \pm \dfrac{|\vec{F}_y^{(i)}(t)|^2}{|\vec{F}_{TOT}^{(i)}(t)|^2} \right )
\end{cases}
\end{aligned}
\label{Eq:8}
\end{equation}

dove i pedici x ed y distinguono la direzione di cui si sta valutando la probabilità, mentre gli apici $\pm$ specificano rispettivamente se il verso è quello concorde o discorde alla relativa forza gravitazionale.
\\ Tuttavia, definizioni di questo tipo possono essere valide soltanto nel caso banale in cui si considerino tutti i siti adiacenti come disponibili, perchè esclusivamente in tale configurazione sarebbe garantita la normalizzazione ad 1.
\\ Di conseguenza, nel caso più realistico di occupazione parziale o completa dei posti adiacenti ad un certo individuo, si inizia con l'imposizione di annullamento della probabilità di spostamento nelle direzioni già impegnate. Dunque, si ridistribuiscono i valori, che nel caso normale sarebbero stati assegnati a tali probabilità, affinchè sia conservata più inalterata possibile la dinamica del sistema e allo stesso tempo sia soddisfatta la normalizzazione ad 1 richiesta.
\\ Infatti, ipotizzando che un individuo \textit{i} arbitrario abbia un numero di siti adiacenti liberi pari a \textit{n=2} e che quelli occupati siano uno in direzione x e verso discorde alla forza $\vec{F}_x^{(i)}$ e l'altro in direzione y e verso concorde alla forza $\vec{F}_x^{(i)}$, le probabilità di scelta saranno date da

\begin{equation}
\begin{aligned}
\begin{cases}
P_x^{(i, +)} = \dfrac{1}{4} \left ( 1 + \dfrac{|\vec{F}_x^{(i)}|^2}{|\vec{F}_{TOT}^{(i)}|^2} \right ) + \dfrac{1}{4n} \left ( 1 - \dfrac{|\vec{F}_x^{(i)}|^2}{|\vec{F}_{TOT}^{(i)}|^2} \right ) + \dfrac{1}{4n} \left ( 1 + \dfrac{|\vec{F}_y^{(i)}|^2}{|\vec{F}_{TOT}^{(i)}|^2} \right ) \\
P_x^{(i, -)} = 0 \\
P_y^{(i, +)} = 0 \\
P_y^{(i, -)} = \dfrac{1}{4} \left ( 1 - \dfrac{|\vec{F}_y^{(i)}|^2}{|\vec{F}_{TOT}^{(i)}|^2} \right ) + \dfrac{1}{4n} \left ( 1 - \dfrac{|\vec{F}_x^{(i)}|^2}{|\vec{F}_{TOT}^{(i)}|^2} \right ) + \dfrac{1}{4n} \left ( 1 + \dfrac{|\vec{F}_y^{(i)}|^2}{|\vec{F}_{TOT}^{(i)}|^2} \right )
\end{cases}
\end{aligned}
\label{Eq:9}
\end{equation}

e così soddisfino semplicemente la condizione di normalizzazione ad 1.
\\ Da notare come, similarmente al modello ad automa cellulare descritto in Sezione \ref{Sec:3.1}, per \textit{n=0} è permessa la possibilità di rimanere fermi tra un istante temporale e quello successivo.

\section{Analisi e Risultati}
\label{Sec:4}

\subsection{Distanza di Influenza}
\label{Sec:4.1}

L'analisi temporale dell'osservabile di magnetizzazione, definito in analogia al Modello di Ising 2D come

\begin{equation}
M(t) = \dfrac{1}{N} \sum_{i = 0}^{N} s(i,t)
\label{Eq:10}
\end{equation}

permette lo studio del comportamento medio del sistema nell'arco delle simulazioni.
\\ In questo modo, applicando valori sempre maggiori del parametro di distanza d'influenza $\delta$, è stato possibile osservare come ciò comporti una significativa accelerazione della dinamica. 

\begin{figure}[h]
\centering
\begin{subfigure}{0.8\textwidth}
\includegraphics[width=\linewidth]{Immagini/magn_graph_d1.png}
\end{subfigure}
\begin{subfigure}{0.8\textwidth}
\includegraphics[width=\linewidth]{Immagini/magn_graph_d2.png}
\end{subfigure}
\end{figure}
\begin{figure}
\ContinuedFloat
\centering
\begin{subfigure}{0.8\textwidth}
\includegraphics[width=\linewidth]{Immagini/magn_graph_d4.png}
\end{subfigure}
\begin{subfigure}{0.8\textwidth}
\includegraphics[width=\linewidth]{Immagini/magn_graph_d8.png}
\end{subfigure}
\caption{\textit{Grafici riguardanti l'evoluzione temporale della magnetizzazione per quattro valori di distanza d'influenza (in ordine dall'alto verso il basso: $\delta$ = 1, 2, 4, 8) in simulazioni da 500 steps temporali ciascuna.}}
\label{Fig:9}
\end{figure}

Infatti, gli esempi in Figura \ref{Fig:9} riportano chiaramente che il tempo necessario al raggiungimento di una delle due distribuzioni stazionarie ("Tutti +1" e "Tutti -1") si riduce sensibilmente all'aumentare di $\delta$.
\\ Inoltre, per analizzare più dettagliatamente la proprietà di $\delta$ di ridurre le tempistiche del sistema sono stati raccolti ulteriori dati riguardanti i tempi di decisione $\tau$, ossia gli intervalli di tempo necessari a ciascun individuo della popolazione di cambiare la propria opinione.

\begin{figure}[h]
\centering
\begin{subfigure}{0.8\textwidth}
\includegraphics[width=\linewidth]{Immagini/time_graph_d1.png}
\end{subfigure}
\begin{subfigure}{0.8\textwidth}
\includegraphics[width=\linewidth]{Immagini/time_graph_d2.png}
\end{subfigure}
\end{figure}
\begin{figure}
\ContinuedFloat
\centering
\begin{subfigure}{0.8\textwidth}
\includegraphics[width=\linewidth]{Immagini/time_graph_d4.png}
\end{subfigure}
\begin{subfigure}{0.8\textwidth}
\includegraphics[width=\linewidth]{Immagini/time_graph_d8.png}
\end{subfigure}
\caption{\textit{Illustrazioni delle distribuzioni $\phi(\tau)$ dei tempi di decisione $\tau$, ottenute dalle medesime simulazioni, la cui magnetizzazione è riportata in Figura \ref{Fig:9}, assieme al relativo fit di power law in rosso, in scala logaritmica.}}
\label{Fig:10}
\end{figure}

\bigskip  \bigskip  \bigskip  \bigskip
La loro rappresentazione grafica, in Figura  \ref{Fig:10}, ha evidenziato un andamento tipico di una legge a potenza, come ha dimostrato il fit rispetto

\begin{equation}
f(\tau) = c \ \tau ^{a}
\label{Eq:11}
\end{equation}

con \textit{c} e \textit{a} costanti, i cui valori stimati sono nella seguente tabella.

\begin{center}
\begin{tabular}{ |c|c|c|c| } 
\hline
 Simulazioni & \textit{c} & \textit{a} \\
\hline
\multirow{4}{4em}{$\delta$ = 1 \\$\delta$ = 2 \\$\delta$ = 4 \\$\delta$ = 8 } & 0.0162 $\pm$ 0.0009 & -0.35 $\pm$ 0.01 \\ 
& 0.048 $\pm$ 0.002 & -0.61 $\pm$ 0.01 \\ 
& 0.140 $\pm$ 0.004 & -0.92 $\pm$ 0.03 \\ 
& 0.36 $\pm$ 0.01 &  -1.27 $\pm$ 0.07 \\
\hline
\end{tabular}
\end{center}

All'aumentare di $\delta$ è possibile osservare una crescita in valore assoluto sia del coefficiente moltiplicativo \textit{c} sia dell'esponente \textit{a}, che rende la distribuzione $\phi(\tau)$ dei tempi di decisione $\tau$ del sistema sempre più piccata verso lo zero.


\subsection{Influenza delle Condizioni Iniziali}
\label{Sec:4.2}

Uno degli scopi principali di questa tipolologia di modelli di \textit{opinion dynamics} è quello di comprendere le eventuali correlazioni tra condizioni iniziali e distribuzioni stazionarie del sistema.
\\ A tal fine, è stata fatta variare la concentrazione iniziale $c_{-}^{in} = (\nicefrac{N_{-}^{in}}{N})$ di agenti d'opinione -1 nella popolazione in esame e per ciascuno dei valori selezionati sono state svolte 500 simulazioni, così da avere un campione statistico valido per la stima della probabilità di osservazione dei due stati stazionari noti.

\begin{figure}[h]
\centering
\includegraphics[width=0.7\linewidth]{Immagini/random_prob_graph.png}
\caption{\textit{Rappresentazione grafica dei dati e dei fit relativi alle probabilità di osservazione degli stati stazionari ("Tutti +1" e "Tutti -1").}}
\label{Fig:11}
\end{figure}

Come è possibile visualizzare in Figura \ref{Fig:10}, l'andamento di queste probabilità è risultato seguire molto bene quello caratteristico di una tangente iperbolica, opportunamente traslata e modulata nel seguente modo

\begin{equation}
f(c_{-}^{in}) = A \cdot \tanh(k(x+\mu))+\varphi
\label{Eq:12}
\end{equation}

con \textit{A, k, $\mu$ e $\varphi$}, rispettivamente, coefficienti di ampiezza, modulazione, dislocamento orizzontale e verticale.
\\ L'operazione di \textit{fitting} ai dati secondo la forma funzionale appena presentata ha condotto alla stima dei suoi parametri, i quali sono stati riportati nella tabella a fine paragrafo, ma anche ad un valore del coefficiente di determinazione $R^2 = 0.999927$, molto prossimo a quello ottimale (ossia 1).

\medskip
\begin{center}
\begin{tabular}{ |c|c|c|c| } 
\hline
\ Parametri \ & \ Fit su "Tutti +1" \ & \ Fit su "Tutti -1" \ \\
\hline
\multirow{4}{4em}{\quad \textit{A} \\ \quad \textit{k} \\ \quad \textit{$\mu$} \\ \quad \textit{$\varphi$} } & -0.4999 $\pm$ 0.0004 & 0.4999 $\pm$ 0.0004 \\ 
& 17.3 $\pm$ 0.1 & 17.3 $\pm$ 0.1 \\ 
& -0.5006 $\pm$ 0.0008 & -0.5006 $\pm$ 0.0008 \\ 
& 0.5021 $\pm$ 0.0007 &  0.4979 $\pm$ 0.0007 \\
\hline
\end{tabular}
\end{center}
\medskip

Inoltre, è stato interessante scoprire quanto anche le modalità di inizializzazione della popolazione sull'automa cellulare siano influenti sulle probabilità di osservazione dei due stati stazionari noti.
\\ Infatti, al posto dell'inizializzazione puramente randomica utilizzata per i risultati precedenti ne è stata introdotta una nuova affinchè all'istante iniziale soltanto i primi $(c_{-}^{in} \cdot N)$ individui avessero opinione -1. Per chiarezza se ne è esposto un esempio in Figura \ref{Fig:12}.

\begin{figure}[h]
{\centering\par
\makebox[0 pt]{
\includegraphics[width=0.45\linewidth]{Immagini/Random_Inizialization.png}
\includegraphics[width=0.45\linewidth]{Immagini/Fixed_Inizialization.png}}
\par}
\caption{\textit{Esempio di due simulazioni con differenti modalità di inizializzazione: randomica a sinistra, ed ordinata a destra.}}
\label{Fig:12}
\end{figure}
\medskip

Ripetendo fedelmente l'analisi descritta all'inizio, si è potuto constatare che tutte le stime dei parametri del \textit{fit} a tangente iperbolica (nella tabella a seguire) sono risultate compatibili a quelle del caso precedente, ad eccezione di \textit{k} che ha riscontrato una diminuizione di un'ordine di grandezza.

\medskip
\begin{center}
\begin{tabular}{ |c|c|c|c| } 
\hline
\ Parametri \ & \ Fit su "Tutti +1" \ & \ Fit su "Tutti -1" \ \\
\hline
\multirow{4}{4em}{\quad \textit{A} \\ \quad \textit{k} \\ \quad \textit{$\mu$} \\ \quad \textit{$\varphi$} } & -0.500 $\pm$ 0.002 & 0.500 $\pm$ 0.002 \\ 
& 4.8 $\pm$ 0.3 & 4.8 $\pm$ 0.3 \\ 
& -0.500 $\pm$ 0.004 & -0.500 $\pm$ 0.004 \\ 
& 0.506 $\pm$ 0.004 &  0.494 $\pm$ 0.004 \\
\hline
\end{tabular}
\end{center}
\medskip

Ciò ha reso gli andamenti delle probabilità d'osservazione delle distribuzioni stazionarie al variare della concentrazione iniziale $c_{-}^{in}$ di agenti con opinione -1 (in Figura \ref{Fig:13}) meno ripide in prossimità del punto di flesso rispetto a quelli in Figura \ref{Fig:11}. Grazie alla valutazione del coefficiente di determinazione $R^2 = 0.9967$, anch'essi sono risultati molto compatibili ai dati, per cui sono state eseguite 500 simulazioni ciascuno.

\medskip
\begin{figure}[h]
\centering
\includegraphics[width=0.9\linewidth]{Immagini/fixed_prob_graph.png}
\caption{\textit{Rappresentazione grafica dei dati e dei fit relativi alle probabilità di osservazione degli stati stazionari ("Tutti +1" e "Tutti -1") nel caso particolare di inizializzazione ordinata della popolazione sull'automa cellulare.}}
\label{Fig:13}
\end{figure}
\medskip \bigskip \bigskip \bigskip

\subsection{Visione Parziale}
\label{Sec:4.3}

Per il modello di visione parziale, le cui basi teoriche sono trattate in Sezione \ref{Sec:3.2}, è stata svolta un'analisi del tutto analoga a quella presentata in Sezione \ref{Sec:4.1} riguardo la risposta del sistema a diversi valori della distanza d'influenza $\delta$.
\\ Innanzitutto sono stati valutati, a fine puramente qualitativo, gli andamenti caratteristici dell'osservabile di magnetizzazione al variare del parametro $\eta$ (in Figura \ref{Fig:14}). Similarmente a $\delta$, si è ottenuto che il suo progressivo aumento comporta un raggiungimento dello stato stazionario a tempi mediamente minori.

\begin{figure}[h]
\centering
\begin{subfigure}{0.95\textwidth}
\includegraphics[width=\linewidth]{Immagini/partial_magn_graph_v1.png}
\end{subfigure}
\begin{subfigure}{0.95\textwidth}
\includegraphics[width=\linewidth]{Immagini/partial_magn_graph_v2.png}
\end{subfigure}
\end{figure}
\begin{figure}
\ContinuedFloat
\centering
\begin{subfigure}{0.95\textwidth}
\includegraphics[width=\linewidth]{Immagini/partial_magn_graph_v4.png}
\end{subfigure}
\begin{subfigure}{0.95\textwidth}
\includegraphics[width=\linewidth]{Immagini/partial_magn_graph_v8.png}
\end{subfigure}
\caption{\textit{Grafici riguardanti l'evoluzione temporale della magnetizzazione per quattro valori di distanza d'influenza (in ordine dall'alto verso il basso e da sinistra a destra: $\eta$ = 1, 2, 4, 8) in simulazioni da 300 steps temporali ciascuna.}}
\label{Fig:14}
\end{figure}

\bigskip  \bigskip  \bigskip  
Questo risultato è stato confermato da un punto di vista più quantitativo tramite l'acquisizione dei dati relativi ai tempi di decisione $\tau$ del sistema. In particolare, la loro distribuzione $\phi(\tau)$ si è rivelata seguire l'andamento tipico delle leggi a potenza, tranne che per il caso specifico di $\eta = 1$, per cui invece è risultato essere senza dubbio di forma esponenziale.
\\ Proprio per questo motivo, la funzione di \textit{fit} scelta è stata la seguente

\begin{equation}
f(\tau)=\begin{cases}
          c \ \tau ^{a} \quad &\text{per } \, \eta \neq 1 \\
          c \ e^{a \tau} \quad &\text{per } \, \eta = 1 \\
     \end{cases} .
\label{Eq:13}
\end{equation}

\begin{figure}[h]
\centering
\begin{subfigure}{0.8\textwidth}
\includegraphics[width=\linewidth]{Immagini/partial_time_graph_v1.png}
\end{subfigure}
\begin{subfigure}{0.8\textwidth}
\includegraphics[width=\linewidth]{Immagini/partial_time_graph_v2.png}
\end{subfigure}
\end{figure}
\begin{figure}
\ContinuedFloat
\centering
\begin{subfigure}{0.8\textwidth}
\includegraphics[width=\linewidth]{Immagini/partial_time_graph_v4.png}
\end{subfigure}
\begin{subfigure}{0.8\textwidth}
\includegraphics[width=\linewidth]{Immagini/partial_time_graph_v8.png}
\end{subfigure}
\caption{\textit{Illustrazioni delle distribuzioni $\phi(\tau)$ dei tempi di decisione $\tau$, ottenute dalle medesime simulazioni, la cui magnetizzazione è riportata in Figura \ref{Fig:14}, assieme al relativo fit non-lineare in rosso, in scala logaritmica.}}
\label{Fig:15}
\end{figure}

Le stime dei parametri \textit{c} ed \textit{a} risultanti dai \textit{fit} in Figura \ref{Fig:15} sono state raccolte nella seguente tabella assieme alle proprie incertezze.

\begin{center}
\begin{tabular}{ |c|c|c|c| } 
\hline
 Simulazioni & \textit{c} & \textit{a} \\
\hline
\multirow{4}{4em}{$\eta$ = 1 \\$\eta$ = 2 \\$\eta$ = 4 \\$\eta$ = 8 } & 0.0212 $\pm$ 0.0004 & -0.0194 $\pm$ 0.0005 \\ 
& 0.064 $\pm$ 0.003 & -0.58 $\pm$ 0.02 \\ 
& 0.197 $\pm$ 0.005 & -0.95 $\pm$ 0.02 \\ 
& 0.286 $\pm$ 0.009 &  -1.17 $\pm$ 0.04 \\
\hline
\end{tabular}
\end{center}

In particolare, la loro variazione dimostra che al decrescere di $\eta$ gli agenti abbiano una probabilità sempre meno trascurabile di compiere cambi di opinione in tempi maggiori.


\subsection{Flocking Gravitazionale}
\label{Sec:4.3}

Le regole di evoluzione della dinamica spaziale presentate nella Sezione \ref{Sec:3.3} hanno effetti notevoli sulla dinamica delle opinioni in quanto le due fazioni corrispondenti ai due stati $\pm1$ intraprendono una specie di ''guerra di logoramento'' in cui gli individui si dispongono in ''trincee'' e cercano di far prevalere la propria opinione (Vedi Fig.\ref{Fig:16}). 

\begin{figure}[h]
\centering
\includegraphics[width=0.65\linewidth]{Gravitazione/trincea.jpg}
\caption{\textit{Configurazione caratteristica degli individui in una dinamica di flocking gravitazionale.}}
\label{Fig:16}
\end{figure}

Gli individui nei nuclei di queste trincee, che sono protetti dai vicini della stessa opinione, sono chiaramente difficili da conquistare mentre le parti più esterne, corrispondenti agli individui che si devono ancora aggregare al gruppo, sono più vulnerabili al cambio di opinione. La fine della guerra corrisponde a quando una delle due fazioni ha influenzato tutti gli individui dell'altra fazione, con il conseguente risultato di una distribuzione di opinioni omogenea corrispondente ad una condizione di equilibrio stabile. Il raggiungimento di una delle due possibili condizioni stazionarie è dipendente dal parametro distanza di influenza $\delta$,  in particolare la dinamica è mediamente tanto più rapida quanto è maggiore il valore di $\delta$.\medskip
\\Va tuttavia osservato che la correlazione tra la rapidità del sistema a ragiungere l'equilibrio e i valori della distanza di influenza non è forte come per i due precedenti modelli. Infatti, in una porzione significante di casi, capita che il sistema riesca a raggiungere l'equilibrio solo in tempi molto alti mentre, in un'altra porzione di casi altrettanto significante, il sistema rilassa all'equilibrio in tempi comparabili con quelli degli altri modelli. La distanza $\delta$ è la responsabile della gestione delle proporzioni di queste due classi di casi: per $\delta$ elevati (rispetto a \textit{$\sqrt k$}), il sistema riesce generalmente a raggiungere una condizione di equilibrio prima che il \textit{flocking} di individui con la stessa opinione sia consistente.

\begin{figure}[h]
\centering
\begin{subfigure}{0.95\textwidth}
\centering
\includegraphics[width=0.9\linewidth]{Gravitazione/mag_d1.pdf}
\end{subfigure}
\begin{subfigure}{0.95\textwidth}
\centering
\includegraphics[width=0.9\linewidth]{Gravitazione/mag_d2.pdf}
\end{subfigure}
\end{figure}
\begin{figure}[t]
\ContinuedFloat
\centering
\begin{subfigure}{0.95\textwidth}
\centering
\includegraphics[width=0.9\linewidth]{Gravitazione/mag_d4.pdf}
\end{subfigure}
\begin{subfigure}{0.95\textwidth}
\centering
\includegraphics[width=0.9\linewidth]{Gravitazione/mag_d8.pdf}
\end{subfigure}
\caption{\textit{Andamento temporale della magnetizzazione per diversi valori della distanza di influenza $\delta$.}}
\label{Fig:17}
\end{figure}
Quando questo accade, il modello di \textit{flocking} gravitazionale si comporta similmente agli altri modelli, come mostrato negli esempi riportati in Figura \ref{Fig:17} attraverso l'analisi delle magnetizzazioni M(t). \\
Se invece le distanze $\delta$ in gioco sono piccole, allora questa evenienza è piuttosto rara e il sistema generalmente si pone in ''condizioni di stallo'' (per via della struttura della dinamica) da cui se ne esce solo dopo tempi lughissimi. Le condizioni di stallo, che hanno probabilità di verificarsi proporzionale al valore di $\delta$, sono quindi le dirette responsabili della rottura della legge a potenza nei grafici dei tempi di decisione.  L'analisi dei tempi di decisione $\tau$ mostrata dai \textit{fit} in Figura \ref{Fig:18} eseguiti tramite la funzione introdotta nella Eq.\ref{Eq:11} mette in luce questa rottura. Le stime dei coefficienti \textit{c} e \textit{a} sono presentate nella seguente tabella.

\begin{center}
\begin{tabular}{ |c|c|c|c| } 
\hline
 Simulazioni & \textit{c} & \textit{a} \\
\hline
\multirow{4}{4em}{$\delta$ = 1 \\$\delta$ = 2 \\$\delta$ = 4 \\$\delta$ = 8 }
& 0.182 $\pm$ 0.007 & -1.09 $\pm$ 0.05 \\ 
& 0.394 $\pm$ 0.011 & -1.68 $\pm$ 0.08 \\ 
& 0.265 $\pm$ 0.005 & -1.54 $\pm$ 0.05 \\ 
& 0.40 $\pm$ 0.02 &  -1.24 $\pm$ 0.08 \\
\hline
\end{tabular}
\end{center}

Fatta eccezione per il grafico avente $\delta$ = 8, dove il \textit{fit} ricopre l'intero intervallo di dati, rimane evidente il fatto che per tempi bassi (fino a circa 20 step temporali di simulazione), il fenomeno di \textit{flocking} non è sufficientemente pesante da creare questa rottura e quindi il sistema segue comunque una legge a potenza.\\
\begin{figure}[h]
\centering
\begin{subfigure}{0.925\textwidth}
\includegraphics[width=\linewidth]{Gravitazione/tau1.pdf}
\end{subfigure}
\begin{subfigure}{0.925\textwidth}
\includegraphics[width=\linewidth]{Gravitazione/tau2.pdf}
\end{subfigure}
\end{figure}\\
\begin{figure}[h]
\ContinuedFloat
\centering
\begin{subfigure}[t]{0.925\textwidth}
\includegraphics[width=\linewidth]{Gravitazione/tau4.pdf}
\end{subfigure}
\begin{subfigure}[t]{0.925\textwidth}
\includegraphics[width=\linewidth]{Gravitazione/tau8.pdf}
\end{subfigure}
\caption{\textit{Distribuzione dei tempi di decisione $\tau$ per diversi valori di $\delta$.}}
\label{Fig:18}
\end{figure}\\
In Figura \ref{Fig:19} sono state messe in evidenza le correlazioni della distanza con la velocità del sistema a rilassare all'equilbrio. Per ognuno dei 4 valori di $\delta$ rappresentati, sono state eseguite 25 simulazioni e il tempo soglia è stato scelto pari a 500 step in quanto è un valore ragionevole dopo il quale i tempi possono essere considerati elevati (e quindi il fenomeno di \textit{flocking} sufficientemente consolidato nella dinamica).
\medskip
\medskip
\\
\begin{figure}[h!!]
\centering
\includegraphics[width=0.85\linewidth]{Gravitazione/Time_distribution_correlation.pdf}
\caption{\textit{Distribuzione delle occorenze di tempi di rilassamento che superano la soglia di 500 step al variare di $\delta$ su un campione di 100 simulazioni. In azzurro le occorrenze per $\delta=1$, in verde per $\delta=2$, in giallo per $\delta=4$ e in rosso per $\delta=8$}.}
\label{Fig:19}
\end{figure}
\\
È interessante analizzare anche un caso in cui il raggiungimento di una condizione di equilibrio stabile può richiedere un numero di step temporali di simulazione molto elevato. 
\begin{figure}[h!]
\centering
\begin{subfigure}[h]{\linewidth}
\centering
\includegraphics[width=\linewidth]{Gravitazione/mag.pdf}
\end{subfigure}
\end{figure}
\begin{figure}
\ContinuedFloat
\centering
\begin{subfigure}[h]{\linewidth}
\centering
\includegraphics[width=\linewidth]{Gravitazione/tau_10078.pdf}
\end{subfigure}
\caption{\textit{Analisi della magnetizzazione e della distribuzione dei tempi di decisione $\tau$ per una simulazione durata 10078 steps.}}
\label{Fig:20}

\end{figure}\\
Qui sopra è stato riportato un esempio in cui si sono studiati la magnetizzazione e i tempi di decisione degli individui in una simulazione durata \textit{10078 steps}.
La stima dei parametri di \textit{fit} del grafico in Figura \ref{Fig:20} sono i seguenti:\\
\begin{center}
\begin{tabular}{ |c|c|c|c| } 
\hline
 Simulazioni & \textit{c} & \textit{a} \\
\hline
\multirow{1}{3em}{$\delta$ = 4 }
& 0.438 $\pm$ 0.010 & -1.48 $\pm$ 0.05 \\ 
\hline
\end{tabular}
\end{center}.


Lo studio della influenza delle condizioni iniziali nella distribuzione di equilibrio del sistema è stato fatto su un campione di 500 simulazioni, eseguite variando la concentrazione $c^{in}_{-}$ di agenti con opinione -1 ogni 100 esecuzioni della simulazione. Il risultato di questa analisi è graficato in Figura \ref{Fig:21}.

I \textit{fit} dei dati sono stati eseguiti attraverso la tangente iperbolica in Eq.\ref{Eq:12} e hanno portato alla stima dei parametri di \textit{fit} riportati in Tabella.

\begin{center}
\begin{tabular}{ |c|c|c|c| } 
\hline
\ Parametri \ & \ Fit su "Tutti +1" \ & \ Fit su "Tutti -1" \ \\
\hline
\multirow{4}{4em}{\quad \textit{A} \\ \quad \textit{k} \\ \quad \textit{$\mu$} \\ \quad \textit{$\varphi$} } 
& -0.4999 $\pm$ 0.0004 & 0.4999 $\pm$ 0.0014 \\ 
& 4.586 $\pm$ 0.011 & 4.586 $\pm$ 0.009 \\ 
& -0.496 $\pm$ 0.003 & -0.496 $\pm$ 0.003 \\ 
& 0.506 $\pm$ 0.003 &  0.494 $\pm$ 0.002 \\
\hline
\end{tabular}
\end{center}

Il coefficiente di determinazione associato $R^2$ = 0.999587 è molto vicino al valore ottimale 1; ciò è indice di un'eccellente compatibilità del \textit{fit} con i dati.
\begin{figure}[h]
\centering
\includegraphics[width=0.9\linewidth]{Gravitazione/fit_grav.pdf}
\caption{\textit{Andamento delle probabilità di osservazione delle condizioni stazionarie nel caso di flocking gravitazionale.}}
\label{Fig:21}
\end{figure}

\subsection{Confronto ai Dati}
\label{Sec:4.5}

In questa sottosezione, si intende mostrare l'attendibilità del modello fin qui introdotto ed analizzato, confrontandone il comportamento con dati riguardanti il supporto o l'opposizione all'adesione della Polonia alla NATO nel lasso temporale compreso tra il 02/1999 e il 03/2022.
\begin{figure}[hb!]
\centering
\includegraphics[width = 0.9\linewidth]{Immagini/poland_magnetization_graph.png}
\caption{\textit{Andamento temporale della magnetizzazione relativa ai dati empirici sul supporto dell'adesione della Polonia alla NATO. }}
\label{Fig:22}
\end{figure}\\
Da notare che la scala di tempi è scelta intenzionalmente fuori scala a causa di una discontinuità dalla raccolta dati tra il 09/2004 e il 02/2014. Tuttavia, questi dati raccolti dal centro di ricerca CBOS \cite{cbos} nella serie \textit{"Polish Public Opinion"} presentano quattro possibili esiti dall'interrogazione delle persone nel campione popolativo, ossia \textit{"Support", "Oppose", "Indifferent"} e \textit{"Don't know/Refusal to answer".}
\\Quindi il calcolo della magnetizzazione, rappresentata in Figura \ref{Fig:22}, è stato possibile assumendo una corrispondenza delle prime due opzioni rispettivamente alle opinioni $\pm1$, e delle ultime due ad un ipotetico valore 0, in modo tale da minimizzarne il loro contributo additivo. Affinchè si ottenga una simulazione che rappresenti qualitativamente l'andamento empirico di magnetizzazione in Figura \ref{Fig:22}, è stato necessario porre la concentrazione iniziale di agenti con opinione -1 a $c_{-}^{in}=0.25$ e il parametro di distanza d'influenza $\delta=2$ (vedi Figura \ref{Fig:23}).
\\ Il primo accorgimento inizializza il sistema in condizioni il più prossime possibili a quelle dei dati; mentre il secondo modula la rapidità di raggiungimento dello stato stazionario, infatti dall'analisi in Sezione \ref{Sec:4.1} si può dedurre che maggiore è il valore di $\delta$, minore sarà il tempo necessario al rilassamento ad uno degli equilibri possibili.

\begin{figure}[h]
\centering
\includegraphics[width = \linewidth]{Immagini/poland_sim_magnetization_graph.png}
\caption{\textit{Andamento temporale della magnetizzazione ottenuta da una simulazione di dinamica delle opinioni secondo il modello su automa cellulare introdotto. }}
\label{Fig:23}
\end{figure}

\section{Implementazione}
\label{Sec:5}

Le simulazioni sono eseguite da un codice \textit{python} strutturato su 3 \textit{files}:

\begin{enumerate}
  \item \textit{"global\_variables.py",}
  \item \textit{"functions.py",}
  \item \textit{"main.py".}
\end{enumerate}

Il primo file, come suggerisce il nome, introduce e inizializza le variabili globali che caratterizzeranno il sistema e la sua dinamica nell'arco della simulazione. Queste sono classificate in categorie a seconda della loro funzionalità. Seguendo fedelmente l'ordine all'interno del file, si trovano:

\begin{itemize}
  \item le impostazioni temporali dell'animazione;
  \item il dimensionamento dell'automa cellulare (implementato da \textit{ambient}, lista di \textit{2D arrays};
  \item la variabile (\textit{setup}) di scelta dello scenario (\textit{"Evoluzione normale", "Flocking gravitazionale"} o \textit{"Visione parziale"});
  \item le variabili booleane di selezione delle analisi da eseguire, di cui si consigliano le seguenti configurazioni:
  \item i parametri di controllo del sistema e della dinamica.
\end{itemize}

Nel secondo file, invece, sono raccolte le funzioni utilizzate per l'esecuzione delle simulazioni. Tra queste le più importanti permettono:

\begin{itemize}
  \item l'evoluzione temporale ("evolve\_norm", "evolve\_grav" e "evolve\_vis", ciascuna delle quali permette lo spostamento e il cambiamento di opinione degli individui della popolazione da un generico istante iniziale a quello successivo nello scenario selezionato tramite \textit{setup});
  \item il calcolo dell'influenza ("influence\_norm" e "influence\_vis", che stimano l'opinione media percepita da un singolo individuo, rispettivamente negli scenari di evoluzione normale e \textit{flocking} gravitazionale e in quello di visione parziale);
  \item l'implementazione dei campi gravitazionali ("local\_density", che organizza in un \textit{array} gli indici di posizione di tutti gli agenti dotati della stessa opinione di quello in esame; "gravity", che calcola modulo e verso delle forze attrattive agenti su ciascun individuo ed "empty\_probs\_grav", che stima il cambiamento nella distribuzione di probabilità di scelta della direzione di spostamento in accordo con quanto asserito in Sezione \ref{Sec:3.3} ).
\end{itemize}

Nel terzo e ultimo file, sono state richiamate le funzioni definite in "\textit{functions.py}" e, a seconda dello scenario desiderato, viene inizializzato l'automa cellulare e fatto evolvere fino al numero di steps temporali fissati. La generazione di numeri pseudo-casuali permette di ottenere risultati diversi ad ogni esecuzione. 

\section{Conclusioni}

Le tre configurazioni analizzate descrivono l'evoluzione temporale dell'opinione in una comunità chiusa. In particolare si è visto che, entro intervalli di tempo sufficientemente lunghi, il sistema raggiunge sempre una delle due condizioni di equilibrio stabile possibili.
Ciò avviene con una rapidità mediamente crescente con il valore della distanza di influenza per i modelli presentati nelle Sezioni \ref{Sec:3.1} e \ref{Sec:3.3}. Tuttavia, il modello di \text{flocking} gravitazionale mostra che, in alcuni casi, la comunità è incapace di riorganizzarsi per trovare un accordo in tempi ragionevoli e il sistema raggiunge una condizione di stallo da cui se ne esce solo dopo tempi estremamente lunghi. L'analisi dei tempi di decisione ha permesso di quantificare la dipendenza del tempo di rilassamento del sistema dalla distanza, interpertando il processo di cambio di opinione del singolo individuo come una transizione del sistema verso la condizione di equilibrio. Quindi, minori sono i tempi di decisione, maggiore è la capacità del sistema di effetturare tante transizioni in pochi step temporali, raggiungendo l'equilibrio rapidamente. Similarmente, per quanto riguarda il modello in Sezione \ref{Sec:3.2}, la rapidità del sistema è risultata mediamente maggiore al crescere dei valori del parametro di visione parziale $\eta$.
Inoltre, è stato scoperto che l'andamento delle probabilità di osservazione delle due distribuzioni stazionarie di opinioni approssima bene una tangente iperbolica la cui pendenza in prossimità del punto di flesso risulta essere sempre minore a seconda dei casi, in ordine: inizializzazione randomica, inizializzazione ordinata e \textit{flocking} gravitazionale.
\\ Il riscontro con i dati raccolti dal centro di ricerca CBOS ha mostrato che, variando opportunamente parametri di controllo, il modello ad automa cellulare riproduce piuttosto bene l'andamento della magnetizzazione relativa ad essi.
\\L'analisi dei modelli si è limitata a lavorare con dimensioni dello spazio, numerosità di agenti e campioni statistici relativamente piccoli a causa dell'elevata complessità computazionale del programma. Un'analisi più approfondita potrebbe tenere in considerazione di lavorare su uno spazio più ampio, con la possibilità di inserire un numero maggiore di agenti, di studiare il comportamento della dinamica con altri valori dei parametri del sistema e di lavorare su campioni statistici più grandi.


\medskip

\printbibliography[title={Bibliografia}]

\end{document}
